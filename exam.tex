\documentclass[a4paper]{report}

\usepackage{cmap} % Улучшенный поиск русских слов в полученном pdf-файле
\usepackage[T2A]{fontenc} % Поддержка русских букв
\usepackage[utf8]{inputenc} % Кодировка utf8
\usepackage[english,russian]{babel} % Языки: русский, английский

\frenchspacing
\usepackage{indentfirst} % Красная строка

\usepackage[left=20mm, right=20mm, top=20mm, bottom=20mm]{geometry}

\usepackage{amsmath}

\usepackage{titlesec}
\titlespacing*{\chapter}{0pt}{-30pt}{8pt}
\titlespacing*{\section}{\parindent}{*4}{*4}
\titlespacing*{\subsection}{\parindent}{*4}{*4}
\titleformat{\chapter}{\LARGE\bfseries}{\thechapter}{20pt}{\LARGE\bfseries}
\titleformat{\section}{\normalfont\bfseries}{\thesection}{5pt}{\normalfont\bfseries}

\begin{document}

\chapter{Случайные события}

\section{Определение пространства элементарных исходов, примеры.
Понятие события (нестрогое), следствие события, невозможное и достоверное событие, примеры.
Операции над событиями.
Сформулировать классическое определение вероятности и доказать его следствия}

\section{Определение пространства элементарных исходов, примеры.
Понятие события (нестрогое).
Сформулировать геометрическое и статистическое определения вероятности.
Достоинства и недостатки этих определений}

\section{Определение пространства элементарных исходов, примеры.
Сформулировать определение сигма-алгебры событий.
Доказать простейшие свойства сигма-алгебры.
Сформулировать аксиоматическое определение вероятности}

\section{Определение пространства элементарных исходов, примеры.
Сформулировать определение сигма-алгебры событий.
Сформулировать аксиоматическое определение вероятности и доказать простейшие свойства вероятности}

\section{Сформулировать определение условной вероятности.
Доказать, что при фиксированном событии $B$ условная вероятность $P(A|B)$ обладает всеми свойствами безусловной вероятности}

\section{Сформулировать определение условной вероятности.
Доказать теорему (формулу) умножения вероятностей.
Привести пример использования этой формулы}

\section{Сформулировать определение пары независимых событий.
Доказать критерий независимости двух событий.
Сформулировать определение попарно независимых событий и событий, независимых в совокупности.
Обосновать связь этих свойств}

\section{Сформулировать определение полной группы событий.
Доказать теоремы о формуле полной вероятности и о формуле Байеса.
Понятия априорной и апостериорной вероятностей}

\section{Сформулировать определение схемы испытаний Бернулли.
Доказать формулу для вычисления вероятности реализации ровно $k$ успехов в серии из $n$ испытаний по схеме Бернулли.
Доказать следствия этой формулы}

\end{document}
