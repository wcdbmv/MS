\documentclass[a4paper]{report}

\usepackage{cmap} % Улучшенный поиск русских слов в полученном pdf-файле
\usepackage[T2A]{fontenc} % Поддержка русских букв
\usepackage[utf8]{inputenc} % Кодировка utf8
\usepackage[english,russian]{babel} % Языки: русский, английский

\frenchspacing
\usepackage{indentfirst} % Красная строка

\usepackage[left=20mm, right=20mm, top=20mm, bottom=20mm]{geometry}

\usepackage{titlesec}
\titlespacing*{\chapter}{0pt}{-30pt}{8pt}
\titlespacing*{\section}{0pt}{*4}{*4}
\titlespacing*{\subsection}{\parindent}{*4}{*4}
\titleformat{\chapter}{\LARGE\bfseries}{\thechapter}{20pt}{\LARGE\bfseries}
\titleformat{\section}{\normalfont\bfseries}{\thesection}{5pt}{\normalfont\bfseries}

\usepackage{amsmath}
\usepackage{amssymb}
\usepackage{commath}
\usepackage{icomma}

%% begin theorem
\usepackage{amsthm}

\makeatletter
\newtheoremstyle{indented}
	{}% measure of space to leave above the theorem
	{}% measure of space to leave below the theorem
	{}% name of font to use in the body of the theorem
	{\parindent}% measure of space to indent
	{\bfseries}% name of head font
	{.}% punctuation between head and body
	{ }% space after theorem head; " " = normal interword space
	{}% header specification (empty for default)
\makeatother

\theoremstyle{indented}

\newtheorem{definition}{Определение}[section]
\newtheorem{example}{Пример}[section]
\newtheorem{theorem}{Теорема}[section]

\makeatletter
\DeclareRobustCommand\bfseriesitshape{%
	\not@math@alphabet\itshapebfseries\relax
	\fontseries\bfdefault
	\fontshape\itdefault
	\selectfont
}
\makeatother

\DeclareTextFontCommand{\textbfit}{\bfseriesitshape}
\DeclareTextFontCommand{\define}{\bfseriesitshape}
%% end theorem


\begin{document}

\chapter{Случайные события}

\section{Определение пространства элементарных исходов, примеры.
Понятие события (нестрогое), следствие события, невозможное и достоверное событие, примеры.
Операции над событиями.
Сформулировать классическое определение вероятности и доказать его следствия}

\begin{definition}
	\label{def:outcomes}
	\define{Элементарные исходы} — мыслимые неделимо исходы случайного эксперимента, из которых в эксперименте происходит ровно один.
\end{definition}

\begin{definition}
	\label{def:sample-space}
	Множество всех элементарных исходов $\Omega$ будем называть \define{пространством элементарных исходов}.
\end{definition}

\begin{example}
	\label{xmp:coin-toss-sample-space}
	Пусть опыт состоит в однократном подбрасывании монеты.
	$\Omega = \{\text{А}, \text{Р}\}$, где $\text{А}$ — аверс, $\text{Р}$ — реверс.

	При двукратном подбрасывании монеты $\Omega = \{\text{АА}, \text{АР}, \text{РА}, \text{РР}\}$.
\end{example}

\begin{example}
	\label{xmp:dice-toss-sample-space}
	При однократном бросании игральной кости возможен любой из шести элементарных исходов $\omega_1, \dots, \omega_6$, где $\omega_i, i=\overline{1, 6}$, означает появление $i$ очков на верхней грани кости.
	$\Omega = \{\omega_i,\quad i = \overline{1, 6}\}$.

	При двукратном бросании игральной кости $\Omega = \{\omega_{ij},\quad i, j = \overline{1, 6}\}$, где $\omega_{ij}$ — исход опыта, при котором сначала выпало $i$, а затем $j$ очков.
\end{example}

\begin{definition}
	\label{def:event}
	\define{Случайное событие} — любое подмножество множества $\Omega$.
\end{definition}

\begin{definition}
	\label{def:consequence-of-the-event}
	Событие $A$ называется \define{следствием события} $B$, если из того, что произошло событие $B$, всегда следует, что произошло $A$, т.~е. $B\subseteq A$.
\end{definition}

\begin{definition}
	\label{def:certain-event}
	Событие, состоящее из всех элементарных исходов, т.~е. событие, которое обязательно происходит в данном опыте, называют \define{достоверным событием}.

	Достоверное событие, как и пространство элементарных исходов, обозначают буквой $\Omega$.
\end{definition}


\begin{definition}
	\label{def:impossible-event}
	Событие, не содержащее ни одного элементарного исхода, т.~е. событие, которое никогда не происходит в данном опыте, называют \define{невозможным событием}.

	Невозможное событие будем обозначать символом $\varnothing$.
\end{definition}

% Примеры самостоятельно

Рассмотрим \define{операции (действия) над событиями}, которые, по существу, совпадают с операциями над подмножествами.
\begin{itemize}
	\item $A \cup B = A + B$ — сумма событий,
	\item $A \cup B = A \cdot B$ — произведение событий,
	\item $A \setminus B = A - B $ — разность событий,
	\item $\overline A = \Omega \setminus A$ — дополнение события.
\end{itemize}

\begin{definition}
	\label{def:classical-probability}

	Пусть
	\begin{enumerate}
		\item $\abs \Omega = N < \infty$,
		\item по условию эксперимента нет оснований предпочесть тот или иной исход остальным (все ЭИ равновозможные).
		\item $\abs A = N_A$ — состоит из $N_A$ элементов.
	\end{enumerate}

	\define{Вероятностью события $A \subseteq \Omega$} называют число
	\begin{equation}
		\label{equ:classical-probability}
		P(A) = \frac{N_A}{N}
	\end{equation}
\end{definition}

\section{Определение пространства элементарных исходов, примеры.
Понятие события (нестрогое).
Сформулировать геометрическое и статистическое определения вероятности.
Достоинства и недостатки этих определений}

\section{Определение пространства элементарных исходов, примеры.
Сформулировать определение сигма-алгебры событий.
Доказать простейшие свойства сигма-алгебры.
Сформулировать аксиоматическое определение вероятности}

\section{Определение пространства элементарных исходов, примеры.
Сформулировать определение сигма-алгебры событий.
Сформулировать аксиоматическое определение вероятности и доказать простейшие свойства вероятности}

\section{Сформулировать определение условной вероятности.
Доказать, что при фиксированном событии $B$ условная вероятность $P(A|B)$ обладает всеми свойствами безусловной вероятности}

\section{Сформулировать определение условной вероятности.
Доказать теорему (формулу) умножения вероятностей.
Привести пример использования этой формулы}

\section{Сформулировать определение пары независимых событий.
Доказать критерий независимости двух событий.
Сформулировать определение попарно независимых событий и событий, независимых в совокупности.
Обосновать связь этих свойств}

\section{Сформулировать определение полной группы событий.
Доказать теоремы о формуле полной вероятности и о формуле Байеса.
Понятия априорной и апостериорной вероятностей}

\section{Сформулировать определение схемы испытаний Бернулли.
Доказать формулу для вычисления вероятности реализации ровно $k$ успехов в серии из $n$ испытаний по схеме Бернулли.
Доказать следствия этой формулы}

\end{document}
