\documentclass[a4paper,oneside,12pt]{extreport}

\usepackage{mmap}
\usepackage[T2A]{fontenc}
\usepackage[utf8]{inputenc}
\usepackage[english,russian]{babel}

\renewcommand{\ttdefault}{PTMono-TLF}

% Текст отчёта следует печатать, соблюдая следующие размеры полей:
% левое — 30 мм, правое — 15 мм, верхнее и нижнее — 20 мм.
\usepackage[left=20mm, right=15mm, top=15mm, bottom=15mm]{geometry}

% \setlength{\parindent}{1.25cm} % Абзацный отступ

\usepackage{setspace}
% \onehalfspacing % Полуторный интервал

\frenchspacing % Равномерные пробелы
\usepackage{indentfirst} % Красная строка

\usepackage{microtype}
\sloppy

\usepackage{titlesec}
\titlespacing*{\chapter}{0pt}{-30pt}{8pt}
\titlespacing*{\section}{\parindent}{*4}{*4}
\titlespacing*{\subsection}{\parindent}{*4}{*4}
\titleformat{\chapter}{\LARGE\bfseries}{\thechapter}{20pt}{\LARGE\bfseries}
\titleformat{\section}{\Large\bfseries}{\thesection}{40pt}{\Large\bfseries}

\usepackage{graphicx}
\usepackage{caption}
\usepackage{float}

\usepackage[unicode,pdftex]{hyperref}
\hypersetup{hidelinks}

%% begin title
\usepackage{wrapfig}

\makeatletter
	\def\vhrulefill#1{\leavevmode\leaders\hrule\@height#1\hfill \kern\z@}
\makeatother
%% end title

%% begin code
\usepackage{listings}
\usepackage{listingsutf8}
\usepackage{xcolor}

\lstset{
	basicstyle=\footnotesize\ttfamily,
	breakatwhitespace=true,
	breaklines=true,
	commentstyle=\color{gray},
	frame=single,
	inputencoding=utf8/koi8-r,
	keywordstyle=\color{blue}\bfseries,
	numbers=left,
	numbersep=5pt,
	numberstyle=\tiny\ttfamily\color{gray},
	showstringspaces=false,
	stringstyle=\color{red},
	tabsize=8
}

%% end code

%% begin theorem
\usepackage{amsthm}

\makeatletter
\newtheoremstyle{indented}
	{}% measure of space to leave above the theorem
	{}% measure of space to leave below the theorem
	{}% name of font to use in the body of the theorem
	{\parindent}% measure of space to indent
	{\bfseries}% name of head font
	{.}% punctuation between head and body
	{ }% space after theorem head; " " = normal interword space
	{}% header specification (empty for default)
\makeatother

\theoremstyle{indented}

\newtheorem{definition}{Определение}[section]
\newtheorem{remark}{Замечание}[section]
%% end theorem


\usepackage{amsmath, amsfonts, amssymb}

\renewcommand\thesection{\arabic{section}}
\renewcommand\thesubsection{\thesection.\arabic{subsection}}


\begin{document}

\include{title}

\paragraph{Цель работы:} построение доверительных интервалов для математического ожидания и дисперсии нормальной случайной величины.

\paragraph{Содержание работы}

\begin{enumerate}
	\item Для выборки объема $n$ из нормальной генеральной совокупности $X$ реализовать в виде программы на ЭВМ
	\begin{itemize}
		\item вычисление точечных оценок $\hat\mu(\vec X_n)$ и $S^2(\vec X_n)$ математического ожидания $MX$ и дисперсии $DX$ соответственно;
		\item вычисление нижней и верхней границ $\underline\mu(\vec X_n)$, $\overline\mu(\vec X_n)$ для $\gamma$-доверительного интервала для математического ожидания $MX$;
		\item вычисление нижней и верхней границ $\underline\sigma^2(\vec X_n)$, $\overline\sigma^2(\vec X_n)$ для $\gamma$-доверительного интервала для дисперсии $DX$;
	\end{itemize}
	\item вычислить $\hat\mu$ и $S^2$ для выборки из индивидуального варианта;
	\item для заданного пользователем уровня доверия $\gamma$ и $N$ – объёма выборки из индивидуального варианта:
	\begin{itemize}
		\item на координатной плоскости $Oyn$ построить прямую $y = \hat\mu(\vec{x_N})$, также графики функций $y = \hat\mu(\vec x_n)$, $y = \underline\mu(\vec x_n)$ и $y = \overline\mu(\vec x_n)$ как функций объема $n$ выборки, где $n$ изменяется от 1 до $N$;
		\item на другой координатной плоскости $Ozn$ построить прямую $z = S^2(\vec{x_N})$, также графики функций $z = S^2(\vec x_n)$, $z = \underline\sigma^2(\vec x_n)$ и $z = \overline\sigma^2(\vec x_n)$ как функций объема $n$ выборки, где $n$ изменяется от 1 до $N$.
	\end{itemize}
\end{enumerate}

\pagebreak
\section{Определение $\gamma$-доверительного интервала}

Пусть $\vec X_n$ — \textit{случайная выборка объема $n$} из \textit{генеральной совокупности $X$} с функцией распределения $F(x;\theta)$, зависящей от параметра $\theta$, значение которого неизвестно.
Предположим, что для параметра $\theta$ построен интервал $\left(\underline\theta(\vec X_n), \overline\theta(\vec X_n)\right)$, где $\underline\theta(\vec X_n)$ и $\overline\theta(\vec X_n)$ являются функциями случайной выборки $\vec X_n$, такими, что выполняется равенство
\begin{equation}
	\mathbf P \left\{ \underline\theta(\vec X_n) < \theta < \overline\theta(\vec X_n) \right\} = \gamma.
\end{equation}
В этом случае интервал $\left(\underline\theta(\vec X_n), \overline\theta(\vec X_n)\right)$ называют {\itshape\bfseries интервальной оценкой} для параметра $\theta$ с {\itshape\bfseries коэффициентом доверия $\gamma$} (или, сокращенно, {\itshape\bfseries $\gamma$-доверительной интервальной оценкой}), а $\underline\theta(\vec X_n)$ и $\overline\theta(\vec X_n)$ соответственно {\itshape\bfseries нижней} и {\itshape\bfseries верхней границами} интервальной оценки.

Интервал $\left(\underline\theta(\vec x_n), \overline\theta(\vec x_n)\right)$ называют {\itshape\bfseries доверительным интервалом} для параметра $\theta$ с коэффициентом доверия $\gamma$ или {\itshape\bfseries $\gamma$-доверительным интервалом}.

\section{Границы $\gamma$-доверительного интервала}

Пусть $\vec X_n$ — случайная выборка объема $n$ из генеральной совокупности $X$, распределенной по нормальному закону с параметрами $\mu$ и $\sigma^{2}$.

\subsection{Оценка для математического ожидания}

\begin{align}
	\underline\mu(\vec X_n) &= \overline X -\frac{S(\vec X_n)}{\sqrt n}t_{1-\alpha}(n-1),\\
	\overline\mu(\vec X_n)  &= \overline X +\frac{S(\vec X_n)}{\sqrt n}t_{1-\alpha}(n-1),
\end{align}
где $\overline X$ — оценка мат. ожидания, $n$ — число опытов, $S(\vec X_n)$ — точечная оценка дисперсии случайной выборки $\vec X_n$, $t_{1-\alpha}(n-1)$ — квантиль уровня $1-\alpha$ для распределения Стьюдента с $n-1$ степенями свободы, $\alpha$ — величина, равная $\displaystyle \frac{(1-\gamma)}2$.

\subsection{Оценка для дисперсии}

\begin{align}
	\underline\sigma^2(\vec X_n) & = \frac{S(\vec X_n)(n-1)}{\chi_{1-\alpha}^2(n-1)},\\
	\overline\sigma^2(\vec X_n)  & = \frac{S(\vec X_n)(n-1)}{\chi_{\alpha}^2(n-1)},
\end{align}
где: $n$ — объем выборки, $\chi_{\alpha}^2(n-1)$ — квантиль уровня $\alpha$ для распределения $\chi^{2}$ с $n-1$ степенями свободы, $\alpha$ — величина, равная $\frac{(1-\gamma)}2$.

\pagebreak
\section{Текст программы}

\lstinputlisting[language=Matlab]{../lab2.m}

\pagebreak
\section{Результаты расчётов}

\begin{equation*}
	\begin{aligned}
		&\hat\mu                      &= & &-7,660917; \\
		&S^2                          &= & & 0,777892; \\
		&\underline\mu(\vec X_n)      &= & &-7,794389; \\
		&\overline\mu(\vec X_n)       &= & &-7,527445; \\
		&\underline\sigma^2(\vec X_n) &= & &0,636386; \\
		&\overline\sigma^2(\vec X_n)  &= & &0,976353.
	\end{aligned}
\end{equation*}

\begin{figure}[H]
	\centering
	\includegraphics[width=0.9\linewidth]{inc/img/figure1}
	\caption{График оценки математического ожидания}
\end{figure}

\begin{figure}[H]
	\centering
	\includegraphics[width=0.9\linewidth]{inc/img/figure2}
	\caption{График оценки дисперсии}
\end{figure}

\end{document}
