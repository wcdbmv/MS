\documentclass[a4paper,oneside,12pt]{extreport}

\usepackage{mmap}
\usepackage[T2A]{fontenc}
\usepackage[utf8]{inputenc}
\usepackage[english,russian]{babel}

\renewcommand{\ttdefault}{PTMono-TLF}

% Текст отчёта следует печатать, соблюдая следующие размеры полей:
% левое — 30 мм, правое — 15 мм, верхнее и нижнее — 20 мм.
\usepackage[left=20mm, right=15mm, top=15mm, bottom=15mm]{geometry}

% \setlength{\parindent}{1.25cm} % Абзацный отступ

\usepackage{setspace}
% \onehalfspacing % Полуторный интервал

\frenchspacing % Равномерные пробелы
\usepackage{indentfirst} % Красная строка

\usepackage{microtype}
\sloppy

\usepackage{titlesec}
\titlespacing*{\chapter}{0pt}{-30pt}{8pt}
\titlespacing*{\section}{\parindent}{*4}{*4}
\titlespacing*{\subsection}{\parindent}{*4}{*4}
\titleformat{\chapter}{\LARGE\bfseries}{\thechapter}{20pt}{\LARGE\bfseries}
\titleformat{\section}{\Large\bfseries}{\thesection}{40pt}{\Large\bfseries}

\usepackage{graphicx}
\usepackage{caption}
\usepackage{float}

\usepackage[unicode,pdftex]{hyperref}
\hypersetup{hidelinks}

%% begin title
\usepackage{wrapfig}

\makeatletter
	\def\vhrulefill#1{\leavevmode\leaders\hrule\@height#1\hfill \kern\z@}
\makeatother
%% end title

%% begin code
\usepackage{listings}
\usepackage{listingsutf8}
\usepackage{xcolor}

\lstset{
	basicstyle=\footnotesize\ttfamily,
	breakatwhitespace=true,
	breaklines=true,
	commentstyle=\color{gray},
	frame=single,
	inputencoding=utf8/koi8-r,
	keywordstyle=\color{blue}\bfseries,
	numbers=left,
	numbersep=5pt,
	numberstyle=\tiny\ttfamily\color{gray},
	showstringspaces=false,
	stringstyle=\color{red},
	tabsize=8
}

%% end code

%% begin theorem
\usepackage{amsthm}

\makeatletter
\newtheoremstyle{indented}
	{}% measure of space to leave above the theorem
	{}% measure of space to leave below the theorem
	{}% name of font to use in the body of the theorem
	{\parindent}% measure of space to indent
	{\bfseries}% name of head font
	{.}% punctuation between head and body
	{ }% space after theorem head; " " = normal interword space
	{}% header specification (empty for default)
\makeatother

\theoremstyle{indented}

\newtheorem{definition}{Определение}[section]
\newtheorem{remark}{Замечание}[section]
%% end theorem


\usepackage{amsmath, amsfonts, amssymb}

\renewcommand\thesection{\arabic{section}}
\renewcommand\thesubsection{\thesection.\arabic{subsection}}


\begin{document}

\include{title}

\paragraph*{Вариант 10}

\section*{Задача № 1 (Предельные теоремы теории вероятности)}

Исследователь зафиксировал по одной реализации каждой из независимых случайных величин $X_1, \ldots, X_{200}$.
Известно, что $DX_i \leqslant 4$, $i = \overline{1;200}$.
Оценить вероятность того, что отклонение среднего арифметического этих случайных величин от среднего арифметического их математического ожидания не превосходит $0,2$.

\paragraph{Решение.}
Пусть
\begin{align*}
	Y &= \frac{1}{200}\sum_{i=1}^{200}X_i, \\
	DY &= \frac{1}{40000}\sum_{i=1}^{200}DX_i \leqslant \frac{1}{40000} \cdot 200 \cdot 4 = 0,02.
\end{align*}

Требуется найти $P\{\abs{Y - MY} \leqslant 0,2\} = 1 - P\{\abs{Y- MY} \geqslant 0,2\}$.

Согласно второму неравенству Чебышева, $\displaystyle P\{\abs{Y- MY} \geqslant 0,2\} \leqslant \frac{DY}{0,04} \leqslant 0,5$.

\begin{equation*}
	P\{\abs{Y- MY} \leqslant 0,2\} \geqslant 1 - 0,5 = 0,5.
\end{equation*}

\paragraph{Ответ:} $P\{\abs{Y- MY} \leqslant 0,2\} \geqslant 0,5$.

\pagebreak
\section*{Задача № 2 (Метод моментов)}

С использованием метода моментов для случайной выборки $\vec X = (X_1, \ldots, X_n)$ из генеральной совокупности $X$ найти точечные оценки указанных параметров заданного закона распределения $\displaystyle f_X(x) = \frac{x^\theta}{\Gamma(\theta + 1)}e^{-x}, x>0$.

\paragraph{Решение.}
\begin{gather*}
	\Gamma(\theta) = \int_0^\infty t^{\theta-1}e^{-t}\,\mathrm dt, \quad \Gamma(\theta + 1) = \theta\Gamma(\theta), \\
	m_1 = MX = \int_0^\infty \frac{x^\theta x}{\Gamma(\theta + 1)}e^{-x} \, \mathrm dx = \frac{1}{\Gamma(\theta + 1)}\int_0^\infty x^{\theta+1}e^{-x}\,\mathrm dx = \frac{\Gamma(\theta + 2)}{\Gamma(\theta + 1)} = \theta + 1, \\
	\hat\mu_1 = \overline x \quad\Rightarrow\quad \theta + 1 = \overline x \quad\Rightarrow\quad \theta = \overline x - 1.
\end{gather*}

\paragraph{Ответ:} $\theta = \overline x - 1$.


\pagebreak
\section*{Задача № 3 (Метод максимального правдоподобия)}

С использованием метода максимального правдоподобия для случайной выборки $\vec X = (X_1, \ldots, X_n)$ из генеральной совокупности $X$ найти точечные оценки параметров заданного закона распределения.
Вычислить выборочные значения найденных оценок для выборки $\vec x_5 = (x_1, \ldots, x_5)$.
$f_X(x) = 2\theta xe^{-\theta x^2}$, $x > 0$, $\vec x_5 = (2, 4, 3, 6, 1)$.

\paragraph{Решение.}
\begin{gather*}
	L(\vec x_5, \theta) = \prod_{i=1}^5 f_X(x_i) = \prod_{i=1}^5 \left(2\theta x_ie^{-\theta x_i^2}\right) = 2^5\theta^5 \prod_{i=1}^5 x_i \cdot \exp\left(\sum_{i=1}^5(-\theta x_i^2)\right). \\
	\ln L(\vec x_5, \theta) = 5\ln2 + 5\ln\theta + \sum_{i=1}^5\ln x_i - \theta\sum_{i=1}^5 x_i^2. \\
	\frac{\partial \ln L(\vec x_5, \theta)}{\partial\theta} = \frac 5\theta - \sum_{i=1}^5 x_i^2 \quad\Rightarrow\quad \hat\theta = \frac{5}{\sum_{i=1}^5 x_i^2} = \frac 5{4 + 16 + 9 + 36 + 1} = \frac 5{66}.
\end{gather*}

\paragraph{Ответ:} $\displaystyle \frac 5{66}$.

\pagebreak
\section*{Задача № 4 (Доверительные интервалы)}

После обработки $n = 8$ результатов независимых наблюдений нормально распределённой случайной величины $X$ получено значение $\sigma^2(\vec X_n) = 5,75$ смещённой оценки выборочной дисперсии.
С какой вероятностью можно гарантировать выполнение неравенства $\vec X_n - 6,2 < MX < \vec X_n + 6,2$?

\paragraph{Решение.}
\begin{multline*}
	P\{\vec X_n - 6,2 < MX < \vec X_n + 6,2\} = P\{\vec X_n - MX - 6,2 < 0 < \vec X_n - MX + 6,2\} = \\
	= P\left\{\abs{\vec X_n - MX} < 6,2\right\} = 1 -P\left\{\abs{\vec X_n - MX} \geqslant 6,2\right\} \geqslant 1 - \frac1n\frac{DX}{6,2^2} = 1 - \frac1{64}\cdot\frac{5,75}{38,44} = 0,997.
\end{multline*}

\paragraph{Ответ:} $0,997$.

\end{document}
