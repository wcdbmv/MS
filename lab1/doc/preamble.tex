\usepackage{mmap}
\usepackage[T2A]{fontenc}
\usepackage[utf8]{inputenc}
\usepackage[english,russian]{babel}

\renewcommand{\ttdefault}{PTMono-TLF}

% Текст отчёта следует печатать, соблюдая следующие размеры полей:
% левое — 30 мм, правое — 15 мм, верхнее и нижнее — 20 мм.
\usepackage[left=20mm, right=15mm, top=15mm, bottom=15mm]{geometry}

% \setlength{\parindent}{1.25cm} % Абзацный отступ

\usepackage{setspace}
% \onehalfspacing % Полуторный интервал

\frenchspacing % Равномерные пробелы
\usepackage{indentfirst} % Красная строка

\usepackage{microtype}
\sloppy

\usepackage{titlesec}
\titlespacing*{\chapter}{0pt}{-30pt}{8pt}
\titlespacing*{\section}{\parindent}{*4}{*4}
\titlespacing*{\subsection}{\parindent}{*4}{*4}
\titleformat{\chapter}{\LARGE\bfseries}{\thechapter}{20pt}{\LARGE\bfseries}
\titleformat{\section}{\Large\bfseries}{\thesection}{40pt}{\Large\bfseries}

\usepackage{graphicx}
\usepackage{caption}
\usepackage{float}

\usepackage[unicode,pdftex]{hyperref}
\hypersetup{hidelinks}

%% begin title
\usepackage{wrapfig}

\makeatletter
	\def\vhrulefill#1{\leavevmode\leaders\hrule\@height#1\hfill \kern\z@}
\makeatother
%% end title

%% begin code
\usepackage{listings}
\usepackage{listingsutf8}
\usepackage{xcolor}

\lstset{
	basicstyle=\footnotesize\ttfamily,
	breakatwhitespace=true,
	breaklines=true,
	commentstyle=\color{gray},
	frame=single,
	inputencoding=utf8/koi8-r,
	keywordstyle=\color{blue}\bfseries,
	numbers=left,
	numbersep=5pt,
	numberstyle=\tiny\ttfamily\color{gray},
	showstringspaces=false,
	stringstyle=\color{red},
	tabsize=8
}

%% end code

%% begin theorem
\usepackage{amsthm}

\makeatletter
\newtheoremstyle{indented}
	{}% measure of space to leave above the theorem
	{}% measure of space to leave below the theorem
	{}% name of font to use in the body of the theorem
	{\parindent}% measure of space to indent
	{\bfseries}% name of head font
	{.}% punctuation between head and body
	{ }% space after theorem head; " " = normal interword space
	{}% header specification (empty for default)
\makeatother

\theoremstyle{indented}

\newtheorem{definition}{Определение}[section]
\newtheorem{remark}{Замечание}[section]
%% end theorem


\usepackage{amsmath, amsfonts, amssymb}

\renewcommand\thesection{\arabic{section}}
\renewcommand\thesubsection{\thesection.\arabic{subsection}}
