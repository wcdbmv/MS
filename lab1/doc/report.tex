\documentclass[a4paper,oneside,12pt]{extreport}

\usepackage{mmap}
\usepackage[T2A]{fontenc}
\usepackage[utf8]{inputenc}
\usepackage[english,russian]{babel}

\renewcommand{\ttdefault}{PTMono-TLF}

% Текст отчёта следует печатать, соблюдая следующие размеры полей:
% левое — 30 мм, правое — 15 мм, верхнее и нижнее — 20 мм.
\usepackage[left=20mm, right=15mm, top=15mm, bottom=15mm]{geometry}

% \setlength{\parindent}{1.25cm} % Абзацный отступ

\usepackage{setspace}
% \onehalfspacing % Полуторный интервал

\frenchspacing % Равномерные пробелы
\usepackage{indentfirst} % Красная строка

\usepackage{microtype}
\sloppy

\usepackage{titlesec}
\titlespacing*{\chapter}{0pt}{-30pt}{8pt}
\titlespacing*{\section}{\parindent}{*4}{*4}
\titlespacing*{\subsection}{\parindent}{*4}{*4}
\titleformat{\chapter}{\LARGE\bfseries}{\thechapter}{20pt}{\LARGE\bfseries}
\titleformat{\section}{\Large\bfseries}{\thesection}{40pt}{\Large\bfseries}

\usepackage{graphicx}
\usepackage{caption}
\usepackage{float}

\usepackage[unicode,pdftex]{hyperref}
\hypersetup{hidelinks}

%% begin title
\usepackage{wrapfig}

\makeatletter
	\def\vhrulefill#1{\leavevmode\leaders\hrule\@height#1\hfill \kern\z@}
\makeatother
%% end title

%% begin code
\usepackage{listings}
\usepackage{listingsutf8}
\usepackage{xcolor}

\lstset{
	basicstyle=\footnotesize\ttfamily,
	breakatwhitespace=true,
	breaklines=true,
	commentstyle=\color{gray},
	frame=single,
	inputencoding=utf8/koi8-r,
	keywordstyle=\color{blue}\bfseries,
	numbers=left,
	numbersep=5pt,
	numberstyle=\tiny\ttfamily\color{gray},
	showstringspaces=false,
	stringstyle=\color{red},
	tabsize=8
}

%% end code

%% begin theorem
\usepackage{amsthm}

\makeatletter
\newtheoremstyle{indented}
	{}% measure of space to leave above the theorem
	{}% measure of space to leave below the theorem
	{}% name of font to use in the body of the theorem
	{\parindent}% measure of space to indent
	{\bfseries}% name of head font
	{.}% punctuation between head and body
	{ }% space after theorem head; " " = normal interword space
	{}% header specification (empty for default)
\makeatother

\theoremstyle{indented}

\newtheorem{definition}{Определение}[section]
\newtheorem{remark}{Замечание}[section]
%% end theorem


\usepackage{amsmath, amsfonts, amssymb}

\renewcommand\thesection{\arabic{section}}
\renewcommand\thesubsection{\thesection.\arabic{subsection}}


\begin{document}

\include{title}

\paragraph{Цель работы:} построение гистограммы и эмпирической функции распределения.

\paragraph{Содержание работы}

\begin{enumerate}
	\item Для выборки объёма $n$ из генеральной совокупности $X$ реализовать в виде программы на ЭВМ
	\begin{itemize}
		\item вычисление максимального значения $M_{\max}$ и минимального значения $M_{\min}$;
		\item размаха $R$ выборки;
		\item вычисление оценок $\hat\mu$ и $S^2$ математического ожидания $MX$ и дисперсии $DX$;
		\item группировку значений выборки в $m = [\log_2 n] + 2$ интервала;
		\item построение на одной координатной плоскости гистограммы и графика функции плотности распределения вероятностей нормальной случайной величины с математическим ожиданием $\hat{\mu}$ и дисперсией $S^2$;
		\item построение на другой координатной плоскости графика эмпирической функции распределения и функции распределения нормальной случайной величины с математическим ожиданием $\hat{\mu}$ и дисперсией $S^2$.
	\end{itemize}
	\item Провести вычисления и построить графики для выборки из индивидуального варианта.
\end{enumerate}

\pagebreak
\section{Формулы для вычисления величин}

\subsection{Минимальное и максимальное значения выборки}
\begin{gather}
	M_{\max} = x_{(n)},\\
	M_{\min} = x_{(1)}.
\end{gather}

\subsection{Размах выборки}
\begin{equation}
	R = M_{\max} - M_{\min}.
\end{equation}

\subsection{Оценки математического ожидания и дисперсии}
\begin{align}
	\hat\mu(\vec x_n) &= \frac 1n \sum_{i=1}^n x_i,\\
	S^2(\vec x_n) &= \frac 1{n-1} \sum_{i=1}^n (x_i-\overline x_n)^2.
\end{align}

\pagebreak
\section{Эмпирическая плотность и гистограмма}

Отрезок $J = [x_{(1)}, x_{(n)}]$, содержащий все выборочные значения, разбивают на $m$ промежутков $J_i$, как правило одинаковой длины $\Delta$.
При этом считают, что каждый промежуток содержит свой левый конец, но лишь последний промежуток содержит и свой правый конец.
Для каждого промежутка $J_i$, $i = \overline{1,m}$, подсчитывают число $n_k$ элементов выборки, попавших в него (при этом $n = n_1 + \ldots + n_m$).

\begin{definition}
	{\itshape\bfseries Эмпирической плотностью распределения}, соответствующей реализации $\vec x_n$ случайной выборки $\vec X_n$ из генеральной совокупности $X$, называют функцию $p_n(x)$, которая во всех точках интервала $J_i, i=\overline{1,m}$, принимает значение $\frac{n_i}{n\Delta}$, а вне интервала $J$ равна нулю, т.~е.
	\begin{equation}
		p_n(x) = \begin{cases}
			\frac{n_{i}}{n\Delta}, & x\in J_{i};\\
			0,                     & x\notin J.
		\end{cases}
	\end{equation}
\end{definition}

График функции $p_n(x)$, представляющий собой кусочно постоянную функцию на промежутке $J = [z_{(1)}, z_{(m)}]$, называют {\itshape\bfseries гистограммой} (см. рис. \ref{img:bar-chart}).

\begin{figure}[H]
	\centering
	\includegraphics[width=0.6\linewidth]{inc/img/bar-chart}
	\caption{}
	\label{img:bar-chart}
\end{figure}

\pagebreak
\section{Эмпирическая функция распределения}

\begin{definition}
	{\itshape\bfseries Эмпирической функцией распределения} называют скалярную функцию $F_n(x)$, которая определена для любого $x\in\mathbb R$ следующим образом:
	\begin{equation}
		F_n = \frac{n(x)}n,
	\end{equation}
	где $n$ — объём выборки.
\end{definition}

Функция $F_n(x)$ обладает всеми свойствами функции распределения.
При этом она кусочно постоянна и изменяется скачками в каждой точке $x_{(i)}$ ($x_{(i)}$ — $i$-й член вариационного ряда).

Если все выборочные значения $x_1, \ldots, x_n$ различны, то функцию $F_n(x)$ можно записать в следующем виде:
\begin{equation}
	F_n(x) = \begin{cases}
		0,        & x \leqslant x_{(1)};\\
		\frac in, & x_{(i)} < x \leqslant x_{(i+1)}, \; i=\overline{1,n-1};\\
		1,        & x \geqslant x_{(n)},
	\end{cases}
\end{equation}
т.~е. в каждой точке $x_{(i)}$ функция $F_n(x)$ имеет скачок величиной $1/n$.

График функции $F_n(x)$ изображён на рис. \ref{img:F}.

\begin{figure}[H]
	\centering
	\includegraphics[width=0.6\linewidth]{inc/img/F}
	\caption{}
	\label{img:F}
\end{figure}

\begin{remark}
	Функция $F_n(x)$ позволяет любую выборку $(x_1, \ldots, x_n)$ интерпретировать как генеральную совокупность $\widetilde X$, все значения которой равновероятны, т.~е.
	\begin{equation}
		\mathbf P\{\widetilde X=x_i\}=\frac1n,\quad i=\overline{1,n}.
	\end{equation}
	Это позволяет рассматривать числовые характеристики случайной величины $\widetilde{X}$ как приближённые значения числовых характеристик случайной величины $X$.
\end{remark}

\pagebreak
\section{Текст программы}

\lstinputlisting[language=Matlab]{../lab1.m}

\pagebreak
\section{Результаты расчётов}

\begin{align*}
	M_{\min}          &= -10,11; \\
	M_{\max}          &= -5,2;   \\
	R                 &= 4,91;   \\
	\hat\mu(\vec X_n) &= -7,6609;\\
	S^2(\vec X_n)     &= 0,7779; \\
	m                 &= 8.
\end{align*}

\begin{figure}[H]
	\centering
	\includegraphics[width=0.6\linewidth]{inc/img/figure1}
	\caption{Гистограмма и график функции плотности распределения вероятностей нормальной случайной величины}
\end{figure}

\begin{figure}[H]
	\centering
	\includegraphics[width=0.6\linewidth]{inc/img/figure2}
	\caption{График выборочной функции распределения и функции распределения нормальной случайной величины}
\end{figure}

\end{document}




